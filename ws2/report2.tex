%%%%%%%%%%%%%%%%%%%%%%%%%%%%%%%%%%%%%%%%%
% Structured General Purpose Assignment
% LaTeX Template
%
% Template Name: Anthony
% The template was named after my friend Anthony.
% Strong inspired by Apache Hadoop and Java (programming language)
%
% Author: Ang LEE
%
% Blog: http://angli.me/
%
% Github: https://github.com/leeang/
%
%%%%%%%%%%%%%%%%%%%%%%%%%%%%%%%%%%%%%%%%%

%----------------------------------------------------------------------------------------
%	CONSTANTS
%----------------------------------------------------------------------------------------

\newcommand{\hmwkTitle}{Report \#2}						% Assignment title
\newcommand{\hmwkClass}{Power System Analysis}			% Course name
\newcommand{\hmwkClassTime}{}							% Workshop time
\newcommand{\hmwkClassInstructor}{}						% Tutor name
\newcommand{\hmwkAuthorName}{Ang LEE}					% Student name

\newcommand{\hmwkGraphicsPath}{img/}					% Graphics path
\newcommand{\hmwkCodePath}{code}						% Code path

%----------------------------------------------------------------------------------------
%	TEMPLATE
%----------------------------------------------------------------------------------------

\documentclass{article}

\usepackage{fancyhdr}	% Required for custom headers
\usepackage{lastpage}	% Required to determine the last page for the footer
\usepackage{extramarks} % Required for headers and footers
\usepackage{graphicx}	% Required to insert images
\graphicspath{\hmwkGraphicsPath}

\usepackage{float}
\usepackage{epstopdf}	% Required to insert .eps images
\usepackage{amssymb}
\usepackage{amsmath}
% \usepackage[hidelinks]{hyperref}

% MATLAB syntax highlighting
\usepackage{color}		% Required to define colors
\definecolor{commentColor}{RGB}{34,139,34}
\definecolor{stringColor}{RGB}{160,32,240}
\usepackage{listings}
\lstset{
	inputpath=\hmwkCodePath,
	language=Matlab,
	basicstyle=\footnotesize\ttfamily,
	keywordstyle=\color{blue},
	stringstyle=\color{stringColor},
	commentstyle=\usefont{T1}{pcr}{m}{n}\color{commentColor},
	tabsize = 4,
	breaklines=true,
	showstringspaces=false,
	numbers=left,
	numberstyle=\scriptsize,
	firstnumber=1,
	numberfirstline=true,
	stepnumber=5,
	frame=leftline
}

% change \textbf textbf
\definecolor{bfcolor}{RGB}{221,75,57}
\DeclareTextFontCommand{\textbf}{\bfseries\color{bfcolor}}

% change \texttt color
\definecolor{ttcolor}{RGB}{0,103,179}
\DeclareTextFontCommand{\texttt}{\ttfamily\color{ttcolor}}

% change headings color
\usepackage{sectsty}
\definecolor{sectioncolor}{RGB}{0,102,33}
\sectionfont{\color{sectioncolor}\sffamily}
\definecolor{subsectioncolor}{RGB}{26,131,171}
\subsectionfont{\color{subsectioncolor}\sffamily}
\definecolor{subsubsectioncolor}{RGB}{0,51,102}
\subsubsectionfont{\color{subsubsectioncolor}\sffamily}

% table
\usepackage{tabu}
\usepackage{booktabs}

% Margins
\topmargin=-0.45in
\evensidemargin=-0.3in
\oddsidemargin=-0.3in
\textwidth=7.1in
\textheight=9.0in
\headsep=0.25in

\linespread{1.1}		% Line spacing

% Set up the header and footer
\pagestyle{fancy}
\lhead{\hmwkTitle} % Header Left 
\chead{} % Header Center
\rhead{\hmwkClass} % Header Right
\lfoot{628305 Fei CAO, 629636 Qi SUN, 631317 Ang LI} % Footer Left
\cfoot{} % Footer Center
\rfoot{Page\ \thepage\ of\ \pageref{LastPage}} % Footer Right
\renewcommand\headrulewidth{0.4pt} % Size of the header rule
\renewcommand\footrulewidth{0.4pt} % Size of the footer rule

\setlength\parindent{0pt} % Removes all indentation from paragraphs

%----------------------------------------------------------------------------------------
%	Problem and Section
%----------------------------------------------------------------------------------------

\newenvironment{homeworkProblem}[1]{
	\section*{#1}
	}{
}
\newenvironment{homeworkSection}[1]{
	\subsection*{#1}
	}{
}
\newcommand{\problemAnswer}[1]{
	\noindent\framebox[\columnwidth][c]{
		\begin{minipage}{0.98\columnwidth}
			#1
		\end{minipage}
	}
}

%----------------------------------------------------------------------------------------
%	Document
%----------------------------------------------------------------------------------------

\begin{document}

\section*{Pre-workshop Calculations}

\subsection*{1.}
\begin{equation}\label{eq1}
\Big( 2 P_{D_{cr}} (G_{line} - \beta B_{line}) - |Y_{line}|^2 |V_1|^2 \Big)^2 - 4 |Y_{line}|^2 P_{D_{cr}}^2 (1 + \beta^2) = 0
\end{equation}

We have known
\begin{align*}
Z_{line} = (0.01 + j0.5)
\Longrightarrow Y_{line} = 0.04 - j1.9992
\Longrightarrow
\begin{cases}
G_{line} = 0.04 \text{ pu}\\
B_{line} = -1.9992 \text{ pu}
\end{cases}
\end{align*}
\begin{align*}
|V_1| &= 1 \text{ pu}\\
\end{align*}

For known $\beta$, Eq. \ref{eq1} can be sovled.
\begin{align*}
P_{D_{cr}} =
\begin{cases}
0.9801999800 \text{ pu} &pf = 1\\
0.4949882502 \text{ pu} &pf = 0.8 \text{ lagging}\\
1.9221529035 \text{ pu} &pf = 0.8 \text{ leading}
\end{cases}
\end{align*}

\begin{equation}\label{eq2}
|V_{2_{cr}}|^2 = \frac{|Y_{line}|^2 |V_1|^2 - 2 P_{D_{cr}} (G_{line} - \beta B_{line})}{2 |Y_{line}|^2}
\end{equation}
Evaluate $|V_{2_{cr}}|$ via Eq. \ref{eq2}
\begin{align*}
|V_{2_{cr}}| =
\begin{cases}
0.700141 \text{ pu} &pf = 1\\
0.556264 \text{ pu} &pf = 0.8 \text{ lagging}\\
1.096169 \text{ pu} &pf = 0.8 \text{ leading}
\end{cases}
\end{align*}

\subsection*{2.}
\begin{equation}\label{eq3}
|V_2|^4 |Y_{line}|^2 + |V_2|^2 \Big(2 P_D (G_{line} - \beta B_{line}) - |Y_{line}|^2 |V_1|^2 \Big) + P_D^2 (1+\beta^2) = 0
\end{equation}
\begin{align*}
P_D = \lambda \cdot \frac{5}{50} \cdot pf = 0.1 \lambda \cdot pf
\end{align*}

We have known
\begin{align*}
pf = 1 &\Longrightarrow \beta = 0\\
\lambda &= 4\\
P_D &= 0.4
\end{align*}

Eq. \ref{eq3} can be solved
\begin{align*}
|V_2| =
\begin{cases}
0.9746 \text{ pu}\\
0.2053 \text{ pu (unstable)}
\end{cases}
\end{align*}

\subsection*{3.}
We have known
\begin{align*}
pf = 0.8 \text{ lagging} &\Longrightarrow \beta = \frac{\sqrt{1 - pf^2}}{pf} = 0.75\\
P_D &= 0.1 \lambda \cdot pf = 0.32
\end{align*}

Eq. \ref{eq3} can be solved
\begin{align*}
|V_2| =
\begin{cases}
0.8343 \text{ pu}\\
0.2398 \text{ pu (unstable)}
\end{cases}
\end{align*}

\subsection*{4.}
We have known
\begin{align*}
pf = 0.8 \text{ leading}&\Longrightarrow \beta = - \frac{\sqrt{1 - pf^2}}{pf} = -0.75\\
P_D &= 0.1 \lambda \cdot pf = 0.32
\end{align*}

Eq. \ref{eq3} can be solved
\begin{align*}
|V_2| =
\begin{cases}
1.0956 \text{ pu}\\
0.1826 \text{ pu (unstable)}
\end{cases}
\end{align*}

%----------------------------------------------------------------------------------------
%	Experiment 1
%----------------------------------------------------------------------------------------

\section*{Experiment 1}

\subsection*{1.}
We write a function \texttt{V\_2 = solve\_eq3(P\_D, beta)} to solve Eq. \ref{eq3}.\\

\begin{tabu} to \textwidth {XXXX}
\toprule
$pf$ &0 &0.8 lagging &0.8 leading\\
\hline
$|V_2|$ &0.974614 pu &0.834335 pu &1.095564pu\\
\bottomrule
\end{tabu}

\subsection*{2. \& 3.}
\begin{figure}[H]
\centering
\includegraphics[width=0.5\linewidth]{e1_q2}
\caption{$P$ - $V$ curve ($pf$ = 1)}
\end{figure}

\subsection*{4.}
\begin{figure}[H]
\centering
\includegraphics[width=0.5\linewidth]{e1_q4}
\caption{$P$ - $V$ curve ($pf$ = 0.8 lagging)}
\end{figure}

\subsection*{5.}
\begin{figure}[H]
\centering
\includegraphics[width=0.5\linewidth]{e1_q5}
\caption{$P$ - $V$ curve ($pf$ = 0.8 leading)}
\end{figure}

\subsection*{6.}
\begin{figure}[H]
\centering
\includegraphics[width=0.8\linewidth]{e1_q6}
\caption{Simulink diagram}
\end{figure}

\subsection*{7.}
\begin{figure}[H]
\centering
\includegraphics[width=\linewidth]{e1_q7}
\caption{Simulink results ($pf$ = 1)}
\end{figure}

\begin{tabu} to \textwidth {XXX}
\toprule
&MATLAB &Simulink\\
\hline
$|V_2|$ &0.974614 pu &0.9749 pu\\
\bottomrule
\end{tabu}

\vspace{9pt}
Results can be considered consistent with each other.

\subsection*{8.}
\begin{figure}[H]
\centering
\includegraphics[width=\linewidth]{e1_q8}
\caption{Simulink results ($pf$ = 0.8 lagging)}
\end{figure}

\begin{tabu} to \textwidth {XXX}
\toprule
&MATLAB &Simulink\\
\hline
$|V_2|$ &0.834335 pu &0.8347 pu\\
\bottomrule
\end{tabu}

\vspace{9pt}
Results can be considered consistent with each other.

\subsection*{9.}
\begin{figure}[H]
\centering
\includegraphics[width=\linewidth]{e1_q9}
\caption{Simulink results ($pf$ = 0.8 leading)}
\end{figure}

\begin{tabu} to \textwidth {XXX}
\toprule
&MATLAB &Simulink\\
\hline
$|V_2|$ &1.095564 pu &1.0957 pu\\
\bottomrule
\end{tabu}

\vspace{9pt}
Results can be considered consistent with each other.

%----------------------------------------------------------------------------------------
%	Experiment 2
%----------------------------------------------------------------------------------------

\section*{Experiment 2}

\subsection*{1.}
\begin{figure}[H]
\centering
\includegraphics[width=0.6\linewidth]{e2_q1}
\caption{$P$ - $V$ curves}
\label{e2_q1}
\end{figure}

\subsection*{3.}
It can be read from Fig. \ref{e2_q1} that $|V_2|$ = 0.8343 pu and $P_D$ = 0.32 pu. Thus, $Q_D = P_D \frac{\sqrt{1 - pf^2}}{pf} = 0.24$ pu.

\begin{tabu} to \textwidth {XXX}
\toprule
&$P$ - $V$ curve &Pre workshop\\
\hline
$|V_2|$ &0.8343 pu &0.8343 pu\\
\hline
$P_D$ &0.32 pu &0.32 pu\\
\hline
$Q_D$ &0.24 pu &0.24 pu\\
\bottomrule
\end{tabu}

\vspace{9pt}
Results can be considered consistent with each other.

\subsection*{4.}
\begin{figure}[H]
\centering
\includegraphics[width=0.8\linewidth]{e2_q4_simulink}
\caption{Simulink diagram}
\end{figure}

\begin{figure}[H]
\centering
\includegraphics[width=\linewidth]{e2_q4_result}
\caption{Simulink results}
\end{figure}

\begin{figure}[H]
\centering
\includegraphics[width=0.5\linewidth]{e2_q4}
\caption{$\beta$ - $V$ curve}
\end{figure}

By using Eq. \ref{eq3} and set $P_D$ as constant value-16MW (16/50 pu). Then we can plot the curve which represents the relationship between $|V_2|$ and $\beta$. Since $|V_2|$ is brought to within $\pm$ 0.02 pu, the maximum $\beta$ is 0.0202. Then we can calculate the power factor of the load. By using power triangle, the compensated reactive power was calculated.

\subsection*{5.}
It can be read from Fig. \ref{e2_q1} that $|V_2|$ = 1.096 pu and $P_D$ = 0.32 pu. Thus, $Q_D = P_D \frac{-\sqrt{1 - pf^2}}{pf} = -0.24$ pu.

\begin{tabu} to \textwidth {XXX}
\toprule
&$P$ - $V$ curve &Pre workshop\\
\hline
$|V_2|$ &1.096 pu &1.0956 pu\\
\hline
$P_D$ &0.32 pu &0.32 pu\\
\hline
$Q_D$ &-0.24 pu &-0.24 pu\\
\bottomrule
\end{tabu}

\vspace{9pt}
Results can be considered consistent with each other.

\subsection*{6.}
As we concluded in the table above, it can be seen that the P-V curve has the highest accuracy. The pre workshop result are identical with a slight difference because of the precision in the calculation process.\\

Furthermore, from the combined P-V curve we can clearly see that the curve with unit power factor is outside of the curve with lagging power factor. Therefore, the Bus 2 voltage in the unit power factor circuit will keep more stable around 1 pu than the lagging power factor circuit when the load complex power was varied.

%----------------------------------------------------------------------------------------
%	Appendix
%----------------------------------------------------------------------------------------

\section*{Appendix}

\newpage
\subsection*{Experiment 1 Task 1}
\lstinputlisting{solve_eq3.m}

%----------------------------------------------------------------------------------------

\end{document}
